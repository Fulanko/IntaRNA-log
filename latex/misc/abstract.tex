\thispagestyle{abstract}
\begin{abstract}\thispagestyle{abstract}
    \setcounter{page}{3}
	Computational prediction of RNA-RNA interactions has become a fast growing topic over the last few years. Many different RNA-RNA interaction prediction tools have been developed, which all use vastly varying methods, ranging from alignment methods over minimum free energy methods to machine learning approaches.
	None of these tools seem to consider the steric 3D constraints of RNA molecules. These constraints could implicate that short intermolecular helices are more likely due to the restricted unpaired regions in stem loops.
	
	This thesis introduces constraints on the intermolecular helix lengths for the prediction tool IntaRNA to improve the overall prediction quality. To understand the composition of RNA structures, more than 3,000 known RNA secondary structures, of any type and organism, were analysed in this thesis. From this analysis, the distribution of helix lengths were learnt, in order to find good starting parameters for the helix length constraints.
	
	In order to evaluate the performance of the developed constraints, I created the IntaRNA benchmark, which is used to compare to the original IntaRNA predictors.
	
	The experiments showed that the new constraint helped to improve the prediction quality of IntaRNA, while reducing the runtime. Further, the results suggest that restricting helices too much has negative effects on the performance.

\end{abstract}\thispagestyle{abstract}
\newpage
\begin{otherlanguage}{german}
    \begin{abstract}\thispagestyle{abstract}
        \setcounter{page}{4}
	Die computer-gestützte Vorhersage von RNA-RNA Interaktionen ist, in den letzten Jahren, zu einer stark wachsenden Thematik geworden. Daher werden immer mehr RNA-RNA Interaktionsvorhersage-Tools entwickelt. Diese Tools verwenden unterschie-dliche Methoden um Interaktionen vorherzusagen. Darunter \emph{alignment} Methoden, \emph{minimum free energy} Methoden und neuerdings auch \emph{machine learning} Ansätze.
	
	Keines dieser Tools scheint jedoch die sterischen 3D-Beschränkungen von RNA-Molekü-len zu berücksichtigen. Genau diese Beschränkungen könnten allerdings bedeuten, dass kurze intermolekulare Helizes wahrscheinlicher sind, da ungepaarte Regionen in sogenannten \emph{stem loops} aufgrund ihrer dreidimensionalen Verdrehung längenbeschränkt sind.	
	
	In dieser Arbeit stelle ich neue Methoden zur Einschränkung der intermolekularen Helixlängen für IntaRNA vor.
	Um die Zusammensetzung von RNA-Strukturen zu verstehen, habe ich über 3000 bekannte RNA-Sekundärstrukturen unterschiedlicher Typen und Organismus analysiert. Dies gab mir eine Intuition zur Verteilung der Helixlängen. Dadurch konnte ich gute Startparameter für die Helixlängen-Beschränkung ermitteln.
	
	Um die Qualität der neuen \emph{prediction modi} zu bewerten, habe ich die IntaRNA-Benchmark erstellt. Diese erlaubt mir Vergleiche zwischen neuen und alten Modi zu ziehen.
	
	Die Experimente haben gezeigt dass die neuen Methoden zur Einschränkung der intermolekularen Helixlängen die Vorhersagequalität von IntaRNA verbessern. Dabei wird gleichzeitig auch die Laufzeit reduziert. Außerdem zeigten die Resultate dass zu kleine Helizes einen negativen Effekt auf die Leistung von IntaRNA haben.
    \end{abstract}
\end{otherlanguage}\thispagestyle{abstract}
\newpage{\pagestyle{abstract}\cleardoublepage}